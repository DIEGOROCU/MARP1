\documentclass[aspectratio=169]{beamer}

% Tema y configuración
\usetheme{Madrid}
\usecolortheme{default}

% Paquetes necesarios
\usepackage[utf8]{inputenc}
\usepackage[spanish]{babel}
\usepackage{amsmath}
\usepackage{amsfonts}
\usepackage{amssymb}
\usepackage{graphicx}
\usepackage{listings}
% Habilitar UTF-8 dentro de lstlisting (acentos, ñ, etc.)
\usepackage{listingsutf8}
\usepackage{xcolor}
\usepackage{verbatim}
% Para dibujos con TikZ
\usepackage{tikz}
\usetikzlibrary{positioning,trees}

% Configuración de listings para código
\lstset{
    basicstyle=\ttfamily\footnotesize,
    keywordstyle=\color{blue}\bfseries,
    commentstyle=\color{green},
    stringstyle=\color{red},
    showstringspaces=false,
    breaklines=true,
    frame=single,
    numbers=left,
    numberstyle=\tiny\color{gray},
    inputencoding=utf8,
    % Mapeo de caracteres españoles para que no falle la compilación
    literate={á}{{\'a}}1 {é}{{\'e}}1 {í}{{\'\i}}1 {ó}{{\'o}}1 {ú}{{\'u}}1
             {Á}{{\'A}}1 {É}{{\'E}}1 {Í}{{\'I}}1 {Ó}{{\'O}}1 {Ú}{{\'U}}1
             {ñ}{{\~n}}1 {Ñ}{{\~N}}1 {ü}{{\"u}}1 {Ü}{{\"U}}1
}

% Información del documento
\title{Hoja 3: Ejercicio 8}
\subtitle{Ejercicios de montículos}
\author{Diego Rodríguez Cubero}
\institute{UCM}
\date{22 de Octubre de 2025}

% Logo (opcional)
% \logo{\includegraphics[height=1cm]{images/logo.png}}

\begin{document}

% Diapositiva de título
\begin{frame}
    \titlepage
\end{frame}

% Tabla de contenidos
\begin{frame}{Contenidos}
    \tableofcontents
\end{frame}

% Sección 1
\section{Enunciado}
\begin{frame}{Enunciado del ejercicio}
    \begin{block}{Ejercicio 8}
        Un montículo k-ario es como un montículo binario pero los nodos internos tienen k hijos en lugar de 2. ¿Cómo se representaría un montículo k-ario en un vector? ¿Cuál es la altura de un montículo k-ario de n elementos en términos de n y k?
    \end{block}
\end{frame}

% Sección 2
\section{Representación en un vector}
\begin{frame}{Idea para la representación en un vector}
    \begin{itemize}
        \item En un montículo k-ario, cada nodo tiene k hijos.
        \item La raíz del montículo se encuentra en la posición 0 del vector.
        \item Las posiciones del vector se llenan siguiendo un recorrido por niveles del árbol.
        \item La altura $h$ tendrá capacidad para $k^{h-1}$ nodos.
    \end{itemize}
\end{frame}
\begin{frame}{Ejemplo de arbol $3$-ario}
    \begin{figure}
        \centering
        \begin{tikzpicture}[level distance=1.5cm,
            level 1/.style={sibling distance=4cm},
            level 2/.style={sibling distance=1cm},
            level 3/.style={sibling distance=1cm},
            level 4/.style={sibling distance=0.4cm}]
            \node[circle,draw] {0}
                child {node[circle,draw] {1}
                    child {node[circle,draw] {4}
                        child {node[circle,draw] {13}}
                        child {node[circle,draw] {14}}
                    }
                    child {node[circle,draw] {5}
                    }
                    child {node[circle,draw] {6}}
                }
                child {node[circle,draw] {2}
                    child {node[circle,draw] {7}}
                    child {node[circle,draw] {8}}
                    child {node[circle,draw] {9}}
                }
                child {node[circle,draw] {3}
                    child {node[circle,draw] {10}}
                    child {node[circle,draw] {11}}
                    child {node[circle,draw] {12}}
                };
        \end{tikzpicture}
        \caption{Ejemplo de un montículo 3-ario representado en un vector}
    \end{figure}
    \[
    V = [0,\ 1,\ 2,\ 3,\ 4,\ 5,\ 6,\ 7,\ 8,\ 9,\ 10,\ 11,\ 12,\ 13,\ 14]
    \]
\end{frame}

\begin{frame}{Representación en un vector}
    \[
    V = [0,\ 1,\ 2,\ 3,\ 4,\ 5,\ 6,\ 7,\ 8,\ 9,\ 10,\ 11,\ 12,\ 13,\ 14]
    \]
    Siendo el elemento $i$-ésimo del vector, el $m$ nodo de la altura $h$ (empezando por la altura $0$ y el nodo $0$ en cada altura):
    \[
        h = min\{d \mid \sum_{j=0}^{d} k^j > i \}
    \]
    \[
        m = i - \sum_{j=0}^{h-1} k^j
    \]
    Por ejemplo, tomando el elemento en la posición $i=11$:
    \[
        h = min\{d \mid \sum_{j=0}^{d} 3^j > 11 \} = 2, \quad
        m = 11 - (1 + 3) = 11 - 4 = 7
    \]
\end{frame}

\begin{frame}{Representación en un vector: Cálculo de padres e hijos}
    Si tenemos un nodo en la posición $i$ del vector, sus hijos y padre se calculan de la siguiente manera, usando induccion para buscar al primer hijo, luego se suman los siguientes hijos que son consecutivos:
    \begin{itemize}
        \item Sea nuestra hipótesis de inducción que la fórmula para encontrar al primer hijo es:
        \[
            \text{hijo}_1(i) = ki + 1
        \]
        \item Caso base $i=0$:
        \[
            \text{hijo}_1(0) = k \cdot 0 + 1 = 1
        \]
        Que es correcto, ya que el primer hijo de la raíz (posición 0) está en la posición 1.
    \end{itemize}
\end{frame}

\begin{frame}{Representación en un vector: Cálculo de padres e hijos (cont.)}
    \begin{itemize}
        \item Paso inductivo: Supongamos que la fórmula es correcta para un nodo en la posición $i$. Entonces, para el nodo en la posición $i+1$:
        \[
            \text{hijo}_1(i+1) = k(i+1) + 1 = ki + k + 1
        \]
        Esto es correcto ya que el nodo $i$ tendrá sus hijos en las posiciones desde $ki + 1$ hasta $ki + k$, por lo que el siguiente nodo $i+1$ tendrá sus hijos desde $ki + k + 1$ hasta $k(i+1) + k$, cumpliendo la fórmula.
        \item El resto de hijos se encuentran en las posiciones:
        \[
            \text{hijo}_j(i) = ki + j, \quad \text{para } j = 1, 2, \ldots, k
        \]
    \end{itemize}
    Así, hemos demostrado que la fórmula para encontrar a los hijos de un nodo en un montículo $k$-ario es correcta.
\end{frame}

\begin{frame}{Representación en un vector: Cálculo de padres e hijos (ejemplo)}
    Por ejemplo, para el nodo en la posición $i=2$ en un montículo $3$-ario:
    \[
        \text{hijo}_1(2) = 3 \cdot 2 + 1 = 7
    \]
    \[
        \text{hijo}_2(2) = 3 \cdot 2 + 2 = 8
    \]
    \[
        \text{hijo}_3(2) = 3 \cdot 2 + 3 = 9
    \]
    Por lo tanto, los hijos del nodo en la posición $2$ están en las posiciones $7$, $8$ y $9$ del vector, que corresponde a los nodos con valores $7$, $8$ y $9$ en nuestro ejemplo anterior.
\end{frame}

\begin{frame}
    \begin{figure}
        \centering
        \begin{tikzpicture}[level distance=1.5cm,
            level 1/.style={sibling distance=4cm},
            level 2/.style={sibling distance=1cm},
            level 3/.style={sibling distance=1cm},
            level 4/.style={sibling distance=0.4cm}]
            \node[circle,draw] {0}
                child {node[circle,draw] {1}
                    child {node[circle,draw] {4}
                        child {node[circle,draw] {13}}
                        child {node[circle,draw] {14}}
                    }
                    child {node[circle,draw] {5}
                    }
                    child {node[circle,draw] {6}}
                }
                child {node[circle,draw] {2}
                    child {node[circle,draw] {7}}
                    child {node[circle,draw] {8}}
                    child {node[circle,draw] {9}}
                }
                child {node[circle,draw] {3}
                    child {node[circle,draw] {10}}
                    child {node[circle,draw] {11}}
                    child {node[circle,draw] {12}}
                };
        \end{tikzpicture}
        \caption{Ejemplo de un montículo 3-ario representado en un vector}
    \end{figure}
    \[
    V = [0,\ 1,\ 2,\ 3,\ 4,\ 5,\ 6,\ 7,\ 8,\ 9,\ 10,\ 11,\ 12,\ 13,\ 14]
    \]
\end{frame}

% Sección 3
\section{Altura del montículo k-ario}
\begin{frame}{Cálculo de la altura del montículo k-ario}
    Si tenemos un montículo $k$-ario con $n$ elementos, la altura $h$ del montículo se puede calcular observando que la cantidad total de nodos hasta la altura $h$ es como mucho:
    \[
        \sum_{j=0}^{h} k^j = \frac{k^{h+1} - 1}{k - 1}
    \]
    Para encontrar la altura en función de $n$ y $k$, necesitamos que esta suma sea al menos $n$, siendo $h$ el menor entero que cumple esta condición:
    \[
        \frac{k^{h+1} - 1}{k - 1} \geq n
    \]
    Despejando $h$, obtenemos:
    \[
        k^{h+1} \geq n(k - 1) + 1 \implies
        h + 1 \geq \log_k(n(k - 1) + 1) \implies
        h \geq \log_k(n(k - 1) + 1) - 1
    \]
    \[
    h = \min \{z \in \mathbb{Z} \mid h \geq \log_k(n(k - 1) + 1) - 1 \}
    \]
\end{frame}

% Sección 4
\section{Conclusión}
\begin{frame}{Conclusión}
    Podemos representar un montículo $k$-ario en un vector llenando los nodos por niveles, y la altura del montículo con $n$ elementos se puede calcular usando la fórmula derivada anteriormente. Además, hemos demostrado cómo calcular las posiciones de los hijos de un nodo dado su índice en el vector.\\
    Todo esto resulta ser muy similar a los montículos binarios, pero adaptado a la estructura $k$-aria sustituyendo el $2$ por $k$ en las fórmulas correspondientes, con algunos ajustes necesarios.
\end{frame}

\end{document}