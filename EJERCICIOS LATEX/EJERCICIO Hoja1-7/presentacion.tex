\documentclass[aspectratio=169]{beamer}

% Tema y configuración
\usetheme{Madrid}
\usecolortheme{default}

% Paquetes necesarios
\usepackage[utf8]{inputenc}
\usepackage[spanish]{babel}
\usepackage{amsmath}
\usepackage{amsfonts}
\usepackage{amssymb}
\usepackage{graphicx}
\usepackage{listings}
\usepackage{xcolor}
\usepackage{verbatim}

% Configuración de listings para código
\lstset{
    basicstyle=\ttfamily\footnotesize,
    keywordstyle=\color{blue}\bfseries,
    commentstyle=\color{green},
    stringstyle=\color{red},
    showstringspaces=false,
    breaklines=true,
    frame=single,
    numbers=left,
    numberstyle=\tiny\color{gray}
}

% Información del documento
\title{Hoja 1: Ejercicio 7}
\subtitle{Ejercicios de análisis amortizado}
\author{Diego Rodríguez Cubero}
\institute{UCM}
\date{\today}

% Logo (opcional)
% \logo{\includegraphics[height=1cm]{images/logo.png}}

\begin{document}

% Diapositiva de título
\begin{frame}
    \titlepage
\end{frame}

% Tabla de contenidos
\begin{frame}{Contenidos}
    \tableofcontents
\end{frame}

% Sección 1: Enunciado del problema
\section{Enunciado del problema}
\begin{frame}{Enunciado del problema}
    \begin{block}{Ejercicio 7}
        Dado un vector $b[0..n-1]$ de enteros mayores que $1$, llamado de bases, un vector $v[0..n-1]$ con $0 \leq v[i] < b[i]$, para todo $i$, representará el valor de un contador expresado en bases $b$, siendo $v[0]$ la cifra menos significativa. El valor del contador viene dado por el sumatorio $\sum_{i=0}^{n-1} \left(v[i] \cdot \prod_{j=0}^{i-1} b[j]\right)$. Implementa la operación $\texttt{incr}$, que incrementa el contador en una unidad, admitiéndose que cuando el contador tenga su valor máximo, $\forall i \in 0..n-1,\ v[i] = b[i] - 1$, al incrementarlo pase a valer $0$, o sea $\forall i \in 0..n-1,\ v[i] = 0$. Demuestra con cualquiera de los métodos vistos en clase, que el coste amortizado de la operación $\texttt{incr}$ está en $O(1)$.
    \end{block}
\end{frame}

% Sección 2: Planteamiento
\section{Planteamiento}
\begin{frame}{Planteamiento}
    \begin{exampleblock}{Definicion}
        Se determina una cota superior de la secuencia entera $T(n)$ para $n$ operaciones. El coste medio por operación es $T(n)/n$.
    \end{exampleblock}
    Conociendo esta definicion, para resolver este problema, debemos calcular $T(n)$ sumando los costes de cada operacion, para despues dividirlo entre $n$ y asi obtener el coste amortizado.\\
    Lo haremos mediante un metodo similar al del contador binario visto en clase, pero adaptado a bases arbitrarias, por lo que calcularemos el coste amortizado de la operacion aproximando la cantidad de veces que se reinicia cada cifra del contador.
\end{frame}
\begin{frame}{Planteamiento}
    \begin{exampleblock}{Definición}
        El contador generalizado en bases $b$ se comporta de la siguiente forma:\\
        Al incrementar, se suma $1$ a la cifra menos significativa, y si se alcanza el máximo en esa posición, se reinicia a $0$ y se incrementa la siguiente cifra, y así sucesivamente.
    \end{exampleblock}
    Por tanto, el coste real de la operacion $\texttt{incr}$ es $1$ mas el numero de cifras que se reinician a $0$.
\end{frame}



\begin{frame}{Ejemplos visuales}
    \begin{table}
        \centering
    {\footnotesize
        \begin{tabular}{|c|c|c|}
            \hline
            	\textbf{Op.} & \textbf{Contador} & \textbf{Coste real} \\
            \hline
            1 & (0,0,0) & 1 \\
            2 & (1,0,0) & 1 \\
            3 & (2,0,0) & 1 \\
            4 & (0,1,0) & 2 \\
            5 & (1,1,0) & 1 \\
            6 & (2,1,0) & 1 \\
            7 & (0,0,1) & 3 \\
            8 & (1,0,1) & 1 \\
            9 & (2,0,1) & 1 \\
            10 & (0,1,1) & 2 \\
            11 & (1,1,1) & 1 \\
            12 & (2,1,1) & 1 \\
            13 & (0,0,0) & 3 \\
            \ldots & \ldots & \ldots \\
            \hline
        \end{tabular}
        }
        \caption{Ejemplo de incrementos en un contador con $b = (3,2,2)$}
    \end{table}
\end{frame}


\begin{frame}{}
    \begin{table}
        \centering
        {\small
        \begin{tabular}{|c|c|c|}
            \hline
            	\textbf{Op.} & \textbf{Contador} & \textbf{Coste real} \\
            \hline
            1 & (0,0,0) & 1 \\
            2 & (1,0,0) & 1 \\
            3 & (0,1,0) & 2 \\
            4 & (1,1,0) & 1 \\
            5 & (0,2,0) & 2 \\
            6 & (1,2,0) & 1 \\
            7 & (0,3,0) & 2 \\
            8 & (1,3,0) & 1 \\
            9 & (0,0,1) & 3 \\
            10 & (1,0,1) & 1 \\
            11 & (0,1,1) & 2 \\
            12 & (1,1,1) & 1 \\
            13 & (0,2,1) & 2 \\
            14 & (1,2,1) & 1 \\
            15 & (0,3,1) & 2 \\
            16 & (1,3,1) & 1 \\
            \ldots & \ldots & \ldots \\
            \hline
        \end{tabular}
        }
        \caption{Ejemplo de incrementos en un contador con $b = (2,4,2)$}
    \end{table}
\end{frame}

% Sección 3: Resolución
\section{Resolución}

\begin{frame}{Resolución: Método del potencial}
    Definimos una función potencial $\Phi(v)$ como el número de cifras distintas de $0$ en el contador $v$. Cada vez que incrementamos, el coste real es igual al número de cifras que se ponen a $0$ más $1$ (la cifra que se incrementa sin reiniciar).
    
    El coste amortizado de una operación es:
    \[
    \hat{c} = c + \Phi(v') - \Phi(v)
    \]
    donde $c$ es el coste real, $v$ el estado antes de la operación y $v'$ después.
\end{frame}

\begin{frame}{Cálculo del coste amortizado}
    Cuando se incrementa el contador:
    \begin{itemize}
        \item Se ponen a $0$ $k$ cifras (las menos significativas), y se incrementa la siguiente ($k$ puede ser $0$ si no hay acarreo).
        \item El coste real es $k+1$.
        \item La función potencial disminuye en $k-1$ (pues se ponen a $0$ $k$ cifras y una pasa de $0$ a $1$).
    \end{itemize}
    Por tanto:
    \[
    \hat{c} = (k+1) + (1-k) = 2
    \]
    Así, el coste amortizado es $2$ para cualquier operación $\texttt{incr}$, es decir, $O(1)$.
\end{frame}


% Sección 4: Conclusión
\section{Conclusión}
\begin{frame}{Conclusión}
    Hemos demostrado que, para un contador en bases arbitrarias, la operación $\texttt{incr}$ tiene coste amortizado $O(1)$ usando el método del potencial. Esto significa que, aunque algunas operaciones puedan requerir reiniciar varias cifras, el coste promedio por operación sigue siendo constante.
\end{frame}

\end{document}