\documentclass[aspectratio=169]{beamer}

% Tema y configuración
\usetheme{Madrid}
\usecolortheme{default}

% Paquetes necesarios
\usepackage[utf8]{inputenc}
\usepackage[spanish]{babel}
\usepackage{amsmath}
\usepackage{amsfonts}
\usepackage{amssymb}
\usepackage{graphicx}
\usepackage{listings}
\usepackage{xcolor}
\usepackage{verbatim}

% Configuración de listings para código
\lstset{
    basicstyle=\ttfamily\footnotesize,
    keywordstyle=\color{blue}\bfseries,
    commentstyle=\color{green},
    stringstyle=\color{red},
    showstringspaces=false,
    breaklines=true,
    frame=single,
    numbers=left,
    numberstyle=\tiny\color{gray}
}

% Información del documento
\title{Hoja 1: Ejercicio 3}
\subtitle{Ejercicios de análisis amortizado}
\author{Diego Rodríguez Cubero}
\institute{UCM}
\date{\today}

% Logo (opcional)
% \logo{\includegraphics[height=1cm]{images/logo.png}}

\begin{document}

% Diapositiva de título
\begin{frame}
    \titlepage
\end{frame}

% Tabla de contenidos
\begin{frame}{Contenidos}
    \tableofcontents
\end{frame}

% Sección 1: Enunciado del problema
\section{Enunciado del problema}
\begin{frame}{Enunciado del problema}
    \begin{block}{Ejercicio 3}
        Se realiza una secuencia de $n$ operaciones sobre una estructura de datos. La operación i-ésima tiene un coste igual a i si i es una potencia de $2$ y, en caso contrario, igual a $1$. Utilizar el método de agregación para determinar el coste amortizado de las operaciones.
    \end{block}
\end{frame}

% Sección 2: Planteamiento
\section{Planteamiento}
\begin{frame}{Planteamiento}
    \begin{exampleblock}{Definicion: Método de agregación}
        Se determina una cota superior de la secuencia entera $T(n)$ para $n$ operaciones. El coste medio por operación es $T(n)/n$.
    \end{exampleblock}
    Conociendo esta definicion, para resolver este problema, debemos calcular $T(n)$ sumando los costes de cada operacion, para despues dividirlo entre $n$ y asi obtener el coste amortizado.
\end{frame}

\begin{frame}{Ejemplo visual}
    \begin{table}
        \centering
        \begin{tabular}{|c|c|}
            \hline
            \textbf{Operación (i)} & \textbf{Coste} \\
            \hline
            1 & 1 \\
            \textbf{2} & \textbf{2} \\
            3 & 1 \\
            \textbf{4} & \textbf{4} \\
            5 & 1 \\
            6 & 1 \\
            7 & 1 \\
            \textbf{8} & \textbf{8} \\
            \ldots & \ldots \\
            \hline
        \end{tabular}
        \caption{Coste de las primeras operaciones}
    \end{table}
\end{frame}

% Sección 3: Resolución
\section{Resolución}

\begin{frame}{Resolución: Método de agregación}
    Primero calculemos $T(n)$:
    \[
    T(n) = \sum_{i=1}^{n} C[i]
    \]
    Donde:
    \[
    C[i] = 
    \begin{cases}
        i & \text{si } i \text{ es potencia de } 2 \\
        1 & \text{en caso contrario}
    \end{cases}
    \]
\end{frame}

\begin{frame}{Separación de la suma}
    Para ello, dividamos la suma en dos partes: las operaciones que son potencias de $2$ y las que no lo son.
    \[
    T(n) = \sum_{\substack{i=1 \\ i \text{ no es potencia de } 2}}^{n} 1 + \sum_{\substack{i=1 \\ i \text{ es potencia de } 2}}^{n} i = \underbrace{P(n)}_{\text{no potencias de } 2} + \underbrace{Q(n)}_{\text{potencias de } 2}
    \]
\end{frame}

\begin{frame}{Desarrollo de los sumatorios}
    Podemos acotar $P(n)$ y $Q(n)$:
    \begin{itemize}
        \item $P(n) \leq n$ \\
        Como mucho haremos $n$ operaciones, de las cuales no todas son potencias de $2$, por lo que el coste total de estas operaciones (todas de coste constante $O(1)$) no puede superar $n$, así obtenemos el coste de todas estas operaciones.
        \item $Q(n) \leq 2n$ \\
        Esta desigualdad se puede probar trivialmente por induccion sobre la serie $\sum_{k=0}^{m} 2^k = 2^{m+1} - 1$. Siendo $m$ el mayor exponente de $2$ tal que $2^m \leq n$. Por lo que el coste total de las operaciones que son potencias de $2$ no puede superar $2n \geq 2^{m+1}$. 
    \end{itemize}
\end{frame}

\begin{frame}{Cota final del coste total}
    Sumando ambas cotas, obtenemos:
    \[
    T(n) = P(n) + Q(n) \leq n + 2n = 3n
    \]
    Por lo tanto, el coste amortizado por operación es:
    \[
    \frac{T(n)}{n} \leq \frac{3n}{n} = 3
    \]
    Así, el coste amortizado de cada operación es $O(1)$.
\end{frame}

% Sección 4: Conclusión
\section{Conclusión}
\begin{frame}{Conclusión}
    En este ejercicio hemos analizado el coste amortizado de una secuencia de operaciones sobre una estructura de datos. Utilizando el método de agregación, hemos llegado a la conclusión de que el coste amortizado por operación es $O(1)$, lo que implica que, a pesar de que algunas operaciones pueden ser costosas, el coste promedio se mantiene constante.
\end{frame}

\end{document}